\section{Virtio Over Fabrics}\label{sec:Virtio Transport Options / Virtio Over Fabrics}

Virtio Over Fabrics (Virtio-oF) enables operations over fabrics that rely
primarily on message passing.

Virtio-oF uses a reliable connection to transmit data. The reliable
connection facilitates communication between entities playing the following roles:

\begin{itemize}
\item A Virtio-oF initiator functions as a Virtio-oF client.
The Virtio-oF initiator sends commands and associated data from the driver
to the Virtio-oF target.
\item A Virtio-oF target functions as a Virtio-oF server.
The Virtio-oF target forwards commands to the device and sends completions
and associated data back to the Virtio-oF initiator.
\end{itemize}

Virtio-oF has the following features:

\begin{itemize}
\item A Virtio-oF target is allowed to be connected by 0 or more Virtio-oF initiators.
\item A Virtio-oF initiator is allowed to connect to a single Virtio-oF target only.
A Virtio-oF device instance is a virtio device that the Virtio-oF initiator is
accessing through the Virtio-oF target.
\item There is a one-to-one mapping between the Virtio-oF queue and the reliable connection.
\item There is one, and only one, Virtio-oF control queue for a Virtio-oF device instance.
The Virtio-oF control queue is used to execute control commands,
for example, to get the Virtio Device ID.
\item There is a one-to-one mapping between virtqueue and Virtio-oF virtqueue
which executes the bulk data transport on virtio devices.
\item The arrival of data on the Virtio-oF queue indicates that a notification has arrived.
\end{itemize}


\subsection{Virtio-oF Qualified Name}\label{sec:Virtio Transport Options / Virtio Over Fabrics / Virtio-oF Qualified Name}
Virtio-oF Qualified Names (VQNs) are used to uniquely describe a Virtio-oF initiator
or a Virtio-oF target for identification.

A VQN is encoded as a string of Unicode characters with the following properties:

\begin{itemize}
\item The encoding is UTF-8 (refer to RFC 3629).
\item The characters dash('-'), dot ('.') and slash('/') are used in formatting.
\item The string is NUL terminated.
\item The maximum name is 256 bytes in length, including the NUL character.
\item There is no strict style limitation.
\end{itemize}
