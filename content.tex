\chapter{Basic Facilities of a Virtio Device}\label{sec:Basic Facilities of a Virtio Device}

A virtio device is discovered and identified by a bus-specific method
(see the bus specific sections: \ref{sec:Virtio Transport Options / Virtio Over PCI Bus}~\nameref{sec:Virtio Transport Options / Virtio Over PCI Bus},
\ref{sec:Virtio Transport Options / Virtio Over MMIO}~\nameref{sec:Virtio Transport Options / Virtio Over MMIO} and \ref{sec:Virtio Transport Options / Virtio Over Channel I/O}~\nameref{sec:Virtio Transport Options / Virtio Over Channel I/O}).  Each
device consists of the following parts:

\begin{itemize}
\item Device status field
\item Feature bits
\item Notifications
\item Device Configuration space
\item One or more virtqueues
\end{itemize}

\section{\field{Device Status} Field}\label{sec:Basic Facilities of a Virtio Device / Device Status Field}
During device initialization by a driver,
the driver follows the sequence of steps specified in
\ref{sec:General Initialization And Device Operation / Device
Initialization}.

The \field{device status} field provides a simple low-level
indication of the completed steps of this sequence.
It's most useful to imagine it hooked up to traffic
lights on the console indicating the status of each device.  The
following bits are defined (listed below in the order in which
they would be typically set):
\begin{description}
\item[ACKNOWLEDGE (1)] Indicates that the guest OS has found the
  device and recognized it as a valid virtio device.

\item[DRIVER (2)] Indicates that the guest OS knows how to drive the
  device.
  \begin{note}
    There could be a significant (or infinite) delay before setting
    this bit.  For example, under Linux, drivers can be loadable modules.
  \end{note}

\item[FAILED (128)] Indicates that something went wrong in the guest,
  and it has given up on the device. This could be an internal
  error, or the driver didn't like the device for some reason, or
  even a fatal error during device operation.

\item[FEATURES_OK (8)] Indicates that the driver has acknowledged all the
  features it understands, and feature negotiation is complete.

\item[DRIVER_OK (4)] Indicates that the driver is set up and ready to
  drive the device.

\item[DEVICE_NEEDS_RESET (64)] Indicates that the device has experienced
  an error from which it can't recover.
\end{description}

The \field{device status} field starts out as 0, and is reinitialized to 0 by
the device during reset.

\drivernormative{\subsection}{Device Status Field}{Basic Facilities of a Virtio Device / Device Status Field}
The driver MUST update \field{device status},
setting bits to indicate the completed steps of the driver
initialization sequence specified in
\ref{sec:General Initialization And Device Operation / Device
Initialization}.
The driver MUST NOT clear a
\field{device status} bit.  If the driver sets the FAILED bit,
the driver MUST later reset the device before attempting to re-initialize.

The driver SHOULD NOT rely on completion of operations of a
device if DEVICE_NEEDS_RESET is set.
\begin{note}
For example, the driver can't assume requests in flight will be
completed if DEVICE_NEEDS_RESET is set, nor can it assume that
they have not been completed.  A good implementation will try to
recover by issuing a reset.
\end{note}

\devicenormative{\subsection}{Device Status Field}{Basic Facilities of a Virtio Device / Device Status Field}

The device MUST NOT consume buffers or send any used buffer
notifications to the driver before DRIVER_OK.

\label{sec:Basic Facilities of a Virtio Device / Device Status Field / DEVICENEEDSRESET}The device SHOULD set DEVICE_NEEDS_RESET when it enters an error state
that a reset is needed.  If DRIVER_OK is set, after it sets DEVICE_NEEDS_RESET, the device
MUST send a device configuration change notification to the driver.

\section{Feature Bits}\label{sec:Basic Facilities of a Virtio Device / Feature Bits}

Each virtio device offers all the features it understands.  During
device initialization, the driver reads this and tells the device the
subset that it accepts.  The only way to renegotiate is to reset
the device.

This allows for forwards and backwards compatibility: if the device is
enhanced with a new feature bit, older drivers will not write that
feature bit back to the device.  Similarly, if a driver is enhanced with a feature
that the device doesn't support, it see the new feature is not offered.

Feature bits are allocated as follows:

\begin{description}
\item[0 to 23, and 50 to 127] Feature bits for the specific device type

\item[24 to 41] Feature bits reserved for extensions to the queue and
  feature negotiation mechanisms

\item[42 to 49, and 128 and above] Feature bits reserved for future extensions.
\end{description}

\begin{note}
For example, feature bit 0 for a network device (i.e.
Device ID 1) indicates that the device supports checksumming of
packets.
\end{note}

In particular, new fields in the device configuration space are
indicated by offering a new feature bit.

To keep the feature negotiation mechanism extensible, it is
important that devices \em{do not} offer any feature bits that
they would not be able to handle if the driver accepted them
(even though drivers are not supposed to accept any unspecified,
reserved, or unsupported features even if offered, according to
the specification.) Likewise, it is important that drivers \em{do
not} accept feature bits they do not know how to handle (even
though devices are not supposed to offer any unspecified,
reserved, or unsupported features in the first place,
according to the specification.) The preferred
way for handling reserved and unexpected features is that the
driver ignores them.

In particular, this is
especially important for features limited to specific transports,
as enabling these for more transports in future versions of the
specification is highly likely to require changing the behaviour
from drivers and devices.  Drivers and devices supporting
multiple transports need to carefully maintain per-transport
lists of allowed features.

\drivernormative{\subsection}{Feature Bits}{Basic Facilities of a Virtio Device / Feature Bits}
The driver MUST NOT accept a feature which the device did not offer,
and MUST NOT accept a feature which requires another feature which was
not accepted.

The driver MUST validate the feature bits offered by the device.
The driver MUST ignore and MUST NOT accept any feature bit that is
\begin{itemize}
\item not described in this specification,
\item marked as reserved,
\item not supported for the specific transport,
\item not defined for the device type.
\end{itemize}

The driver SHOULD go into backwards compatibility mode
if the device does not offer a feature it understands, otherwise MUST
set the FAILED \field{device status} bit and cease initialization.

By contrast, the driver MUST NOT fail solely because a feature
it does not understand has been offered by the device.

\devicenormative{\subsection}{Feature Bits}{Basic Facilities of a Virtio Device / Feature Bits}
The device MUST NOT offer a feature which requires another feature
which was not offered.  The device SHOULD accept any valid subset
of features the driver accepts, otherwise it MUST fail to set the
FEATURES_OK \field{device status} bit when the driver writes it.

The device MUST NOT offer feature bits corresponding to features
it would not support if accepted by the driver (even if the
driver is prohibited from accepting the feature bits by the
specification); for the sake of clarity, this refers to feature
bits not described in this specification, reserved feature bits
and feature bits reserved or not supported for the specific
transport or the specific device type, but this does not preclude
devices written to a future version of this specification from
offering such feature bits should such a specification have a
provision for devices to support the corresponding features.

If a device has successfully negotiated a set of features
at least once (by accepting the FEATURES_OK \field{device
status} bit during device initialization), then it SHOULD
NOT fail re-negotiation of the same set of features after
a device or system reset.  Failure to do so would interfere
with resuming from suspend and error recovery.

\subsection{Legacy Interface: A Note on Feature
Bits}\label{sec:Basic Facilities of a Virtio Device / Feature
Bits / Legacy Interface: A Note on Feature Bits}

Transitional Drivers MUST detect Legacy Devices by detecting that
the feature bit VIRTIO_F_VERSION_1 is not offered.
Transitional devices MUST detect Legacy drivers by detecting that
VIRTIO_F_VERSION_1 has not been acknowledged by the driver.

In this case device is used through the legacy interface.

Legacy interface support is OPTIONAL.
Thus, both transitional and non-transitional devices and
drivers are compliant with this specification.

Requirements pertaining to transitional devices and drivers
is contained in sections named 'Legacy Interface' like this one.

When device is used through the legacy interface, transitional
devices and transitional drivers MUST operate according to the
requirements documented within these legacy interface sections.
Specification text within these sections generally does not apply
to non-transitional devices.

\section{Notifications}\label{sec:Basic Facilities of a Virtio Device
/ Notifications}

The notion of sending a notification (driver to device or device
to driver) plays an important role in this specification. The
modus operandi of the notifications is transport specific.

There are three types of notifications: 
\begin{itemize}
\item configuration change notification
\item available buffer notification
\item used buffer notification. 
\end{itemize}

Configuration change notifications and used buffer notifications are sent
by the device, the recipient is the driver. A configuration change
notification indicates that the device configuration space has changed; a
used buffer notification indicates that a buffer may have been made used
on the virtqueue designated by the notification.

Available buffer notifications are sent by the driver, the recipient is
the device. This type of notification indicates that a buffer may have
been made available on the virtqueue designated by the notification.

The semantics, the transport-specific implementations, and other
important aspects of the different notifications are specified in detail
in the following chapters.

Most transports implement notifications sent by the device to the
driver using interrupts. Therefore, in previous versions of this
specification, these notifications were often called interrupts.
Some names defined in this specification still retain this interrupt
terminology. Occasionally, the term event is used to refer to
a notification or a receipt of a notification.

\section{Device Reset}\label{sec:Basic Facilities of a Virtio Device / Device Reset}

The driver may want to initiate a device reset at various times; notably,
it is required to do so during device initialization and device cleanup.

The mechanism used by the driver to initiate the reset is transport specific.

\devicenormative{\subsection}{Device Reset}{Basic Facilities of a Virtio Device / Device Reset}

A device MUST reinitialize \field{device status} to 0 after receiving a reset.

A device MUST NOT send notifications or interact with the queues after
indicating completion of the reset by reinitializing \field{device status}
to 0, until the driver re-initializes the device.

\drivernormative{\subsection}{Device Reset}{Basic Facilities of a Virtio Device / Device Reset}

The driver SHOULD consider a driver-initiated reset complete when it
reads \field{device status} as 0.

\section{Device Configuration Space}\label{sec:Basic Facilities of a Virtio Device / Device Configuration Space}

Device configuration space is generally used for rarely-changing or
initialization-time parameters.  Where configuration fields are
optional, their existence is indicated by feature bits: Future
versions of this specification will likely extend the device
configuration space by adding extra fields at the tail.

\begin{note}
The device configuration space uses the little-endian format
for multi-byte fields.
\end{note}

Each transport also provides a generation count for the device configuration
space, which will change whenever there is a possibility that two
accesses to the device configuration space can see different versions of that
space.

\drivernormative{\subsection}{Device Configuration Space}{Basic Facilities of a Virtio Device / Device Configuration Space}
Drivers MUST NOT assume reads from
fields greater than 32 bits wide are atomic, nor are reads from
multiple fields: drivers SHOULD read device configuration space fields like so:

\begin{lstlisting}
u32 before, after;
do {
        before = get_config_generation(device);
        // read config entry/entries.
        after = get_config_generation(device);
} while (after != before);
\end{lstlisting}

For optional configuration space fields, the driver MUST check that the
corresponding feature is offered before accessing that part of the configuration
space.
\begin{note}
See section \ref{sec:General Initialization And Device Operation / Device Initialization} for details on feature negotiation.
\end{note}

Drivers MUST
NOT limit structure size and device configuration space size.  Instead,
drivers SHOULD only check that device configuration space is {\em large enough} to
contain the fields necessary for device operation.

\begin{note}
For example, if the specification states that device configuration
space 'includes a single 8-bit field' drivers should understand this to mean that
the device configuration space might also include an arbitrary amount of
tail padding, and accept any device configuration space size equal to or
greater than the specified 8-bit size.
\end{note}

\devicenormative{\subsection}{Device Configuration Space}{Basic Facilities of a Virtio Device / Device Configuration Space}
The device MUST allow reading of any device-specific configuration
field before FEATURES_OK is set by the driver.  This includes fields which are
conditional on feature bits, as long as those feature bits are offered
by the device.

\subsection{Legacy Interface: A Note on Device Configuration Space endian-ness}\label{sec:Basic Facilities of a Virtio Device / Device Configuration Space / Legacy Interface: A Note on Configuration Space endian-ness}

Note that for legacy interfaces, device configuration space is generally the
guest's native endian, rather than PCI's little-endian.
The correct endian-ness is documented for each device.

\subsection{Legacy Interface: Device Configuration Space}\label{sec:Basic Facilities of a Virtio Device / Device Configuration Space / Legacy Interface: Device Configuration Space}

Legacy devices did not have a configuration generation field, thus are
susceptible to race conditions if configuration is updated.  This
affects the block \field{capacity} (see \ref{sec:Device Types /
Block Device / Device configuration layout}) and
network \field{mac} (see \ref{sec:Device Types / Network Device /
Device configuration layout}) fields;
when using the legacy interface, drivers SHOULD
read these fields multiple times until two reads generate a consistent
result.

\section{Virtqueues}\label{sec:Basic Facilities of a Virtio Device / Virtqueues}

The mechanism for bulk data transport on virtio devices is
pretentiously called a virtqueue. Each device can have zero or more
virtqueues\footnote{For example, the simplest network device has one virtqueue for
transmit and one for receive.}.

A virtio device can have maximum of 65536 virtqueues. Each virtqueue is
identified by a virtqueue index. A virtqueue index has a value in the
range of 0 to 65535.

Driver makes requests available to device by adding
an available buffer to the queue, i.e., adding a buffer
describing the request to a virtqueue, and optionally triggering
a driver event, i.e., sending an available buffer notification
to the device.

Device executes the requests and - when complete - adds
a used buffer to the queue, i.e., lets the driver
know by marking the buffer as used. Device can then trigger
a device event, i.e., send a used buffer notification to the driver.

Device reports the number of bytes it has written to memory for
each buffer it uses. This is referred to as ``used length''.

Device is not generally required to use buffers in
the same order in which they have been made available
by the driver.

Some devices always use descriptors in the same order in which
they have been made available. These devices can offer the
VIRTIO_F_IN_ORDER feature. If negotiated, this knowledge
might allow optimizations or simplify driver and/or device code.

Each virtqueue can consist of up to 3 parts:
\begin{itemize}
\item Descriptor Area - used for describing buffers
\item Driver Area - extra data supplied by driver to the device
\item Device Area - extra data supplied by device to driver
\end{itemize}

\begin{note}
Note that previous versions of this spec used different names for
these parts (following \ref{sec:Basic Facilities of a Virtio Device / Split Virtqueues}):
\begin{itemize}
\item Descriptor Table - for the Descriptor Area
\item Available Ring - for the Driver Area
\item Used Ring - for the Device Area
\end{itemize}

\end{note}

Two formats are supported: Split Virtqueues (see \ref{sec:Basic
Facilities of a Virtio Device / Split
Virtqueues}~\nameref{sec:Basic Facilities of a Virtio Device /
Split Virtqueues}) and Packed Virtqueues (see \ref{sec:Basic
Facilities of a Virtio Device / Packed
Virtqueues}~\nameref{sec:Basic Facilities of a Virtio Device /
Packed Virtqueues}).

Every driver and device supports either the Packed or the Split
Virtqueue format, or both.

\subsection{Virtqueue Reset}\label{sec:Basic Facilities of a Virtio Device / Virtqueues / Virtqueue Reset}

When VIRTIO_F_RING_RESET is negotiated, the driver can reset a virtqueue
individually. The way to reset the virtqueue is transport specific.

Virtqueue reset is divided into two parts. The driver first resets a queue and
can afterwards optionally re-enable it.

\subsubsection{Virtqueue Reset}\label{sec:Basic Facilities of a Virtio Device / Virtqueues / Virtqueue Reset / Virtqueue Reset}

\devicenormative{\paragraph}{Virtqueue Reset}{Basic Facilities of a Virtio Device / Virtqueues / Virtqueue Reset / Virtqueue Reset}

After a queue has been reset by the driver, the device MUST NOT execute
any requests from that virtqueue, or notify the driver for it.

The device MUST reset any state of a virtqueue to the default state,
including the available state and the used state.

\drivernormative{\paragraph}{Virtqueue Reset}{Basic Facilities of a Virtio Device / Virtqueues / Virtqueue Reset / Virtqueue Reset}

After the driver tells the device to reset a queue, the driver MUST verify that
the queue has actually been reset.

After the queue has been successfully reset, the driver MAY release any
resource associated with that virtqueue.

\subsubsection{Virtqueue Re-enable}\label{sec:Basic Facilities of a Virtio Device / Virtqueues / Virtqueue Reset / Virtqueue Re-enable}

This process is the same as the initialization process of a single queue during
the initialization of the entire device.

\devicenormative{\paragraph}{Virtqueue Re-enable}{Basic Facilities of a Virtio Device / Virtqueues / Virtqueue Reset / Virtqueue Re-enable}

The device MUST observe any queue configuration that may have been
changed by the driver, like the maximum queue size.

\drivernormative{\paragraph}{Virtqueue Re-enable}{Basic Facilities of a Virtio Device / Virtqueues / Virtqueue Reset / Virtqueue Re-enable}

When re-enabling a queue, the driver MUST configure the queue resources
as during initial virtqueue discovery, but optionally with different
parameters.

\input{split-ring.tex}

\input{packed-ring.tex}

\section{Driver Notifications} \label{sec:Basic Facilities of a Virtio Device / Driver notifications}
The driver is sometimes required to send an available buffer
notification to the device.

When VIRTIO_F_NOTIFICATION_DATA has not been negotiated,
this notification contains either a virtqueue index if
VIRTIO_F_NOTIF_CONFIG_DATA is not negotiated or device supplied virtqueue
notification config data if VIRTIO_F_NOTIF_CONFIG_DATA is negotiated.

The notification method and supplying any such virtqueue notification config data
is transport specific.

However, some devices benefit from the ability to find out the
amount of available data in the queue without accessing the virtqueue in memory:
for efficiency or as a debugging aid.

To help with these optimizations, when VIRTIO_F_NOTIFICATION_DATA
has been negotiated, driver notifications to the device include
the following information:

\begin{description}
\item [vq_index or vq_notif_config_data] Either virtqueue index or device
      supplied queue notification config data corresponding to a virtqueue.
\item [next_off] Offset
      within the ring where the next available ring entry
      will be written.
      When VIRTIO_F_RING_PACKED has not been negotiated this refers to the
      15 least significant bits of the available index.
      When VIRTIO_F_RING_PACKED has been negotiated this refers to the offset
      (in units of descriptor entries)
      within the descriptor ring where the next available
      descriptor will be written.
\item [next_wrap] Wrap Counter.
      With VIRTIO_F_RING_PACKED this is the wrap counter
      referring to the next available descriptor.
      Without VIRTIO_F_RING_PACKED this is the most significant bit
      (bit 15) of the available index.
\end{description}

Note that the driver can send multiple notifications even without
making any more buffers available. When VIRTIO_F_NOTIFICATION_DATA
has been negotiated, these notifications would then have
identical \field{next_off} and \field{next_wrap} values.

\input{shared-mem.tex}

\section{Exporting Objects}\label{sec:Basic Facilities of a Virtio Device / Exporting Objects}

When an object created by one virtio device needs to be
shared with a seperate virtio device, the first device can
export the object by generating a UUID which can then
be passed to the second device to identify the object.

What constitutes an object, how to export objects, and
how to import objects are defined by the individual device
types. It is RECOMMENDED that devices generate version 4
UUIDs as specified by \hyperref[intro:rfc4122]{[RFC4122]}.

\input{admin.tex}

\chapter{General Initialization And Device Operation}\label{sec:General Initialization And Device Operation}

We start with an overview of device initialization, then expand on the
details of the device and how each step is preformed.  This section
is best read along with the bus-specific section which describes
how to communicate with the specific device.

\section{Device Initialization}\label{sec:General Initialization And Device Operation / Device Initialization}

\drivernormative{\subsection}{Device Initialization}{General Initialization And Device Operation / Device Initialization}
The driver MUST follow this sequence to initialize a device:

\begin{enumerate}
\item Reset the device.

\item Set the ACKNOWLEDGE status bit: the guest OS has noticed the device.

\item Set the DRIVER status bit: the guest OS knows how to drive the device.

\item\label{itm:General Initialization And Device Operation /
Device Initialization / Read feature bits} Read device feature bits, and write the subset of feature bits
   understood by the OS and driver to the device.  During this step the
   driver MAY read (but MUST NOT write) the device-specific configuration fields to check that it can support the device before accepting it.

\item\label{itm:General Initialization And Device Operation / Device Initialization / Set FEATURES-OK} Set the FEATURES_OK status bit.  The driver MUST NOT accept
   new feature bits after this step.

\item\label{itm:General Initialization And Device Operation / Device Initialization / Re-read FEATURES-OK} Re-read \field{device status} to ensure the FEATURES_OK bit is still
   set: otherwise, the device does not support our subset of features
   and the device is unusable.

\item\label{itm:General Initialization And Device Operation / Device Initialization / Device-specific Setup} Perform device-specific setup, including discovery of virtqueues for the
   device, optional per-bus setup, reading and possibly writing the
   device's virtio configuration space, and population of virtqueues.

\item\label{itm:General Initialization And Device Operation / Device Initialization / Set DRIVER-OK} Set the DRIVER_OK status bit.  At this point the device is
   ``live''.
\end{enumerate}

If any of these steps go irrecoverably wrong, the driver SHOULD
set the FAILED status bit to indicate that it has given up on the
device (it can reset the device later to restart if desired).  The
driver MUST NOT continue initialization in that case.

The driver MUST NOT send any buffer available notifications to
the device before setting DRIVER_OK.

\subsection{Legacy Interface: Device Initialization}\label{sec:General Initialization And Device Operation / Device Initialization / Legacy Interface: Device Initialization}
Legacy devices did not support the FEATURES_OK status bit, and thus did
not have a graceful way for the device to indicate unsupported feature
combinations.  They also did not provide a clear mechanism to end
feature negotiation, which meant that devices finalized features on
first-use, and no features could be introduced which radically changed
the initial operation of the device.

Legacy driver implementations often used the device before setting the
DRIVER_OK bit, and sometimes even before writing the feature bits
to the device.

The result was the steps \ref{itm:General Initialization And
Device Operation / Device Initialization / Set FEATURES-OK} and
\ref{itm:General Initialization And Device Operation / Device
Initialization / Re-read FEATURES-OK} were omitted, and steps
\ref{itm:General Initialization And Device Operation /
Device Initialization / Read feature bits},
\ref{itm:General Initialization And Device Operation / Device Initialization / Device-specific Setup} and \ref{itm:General Initialization And Device Operation / Device Initialization / Set DRIVER-OK}
were conflated.

Therefore, when using the legacy interface:
\begin{itemize}
\item
The transitional driver MUST execute the initialization
sequence as described in \ref{sec:General Initialization And Device
Operation / Device Initialization}
but omitting the steps \ref{itm:General Initialization And Device
Operation / Device Initialization / Set FEATURES-OK} and
\ref{itm:General Initialization And Device Operation / Device
Initialization / Re-read FEATURES-OK}.

\item
The transitional device MUST support the driver
writing device configuration fields
before the step \ref{itm:General Initialization And Device Operation /
Device Initialization / Read feature bits}.
\item
The transitional device MUST support the driver
using the device before the step \ref{itm:General Initialization
And Device Operation / Device Initialization / Set DRIVER-OK}.
\end{itemize}

\section{Device Operation}\label{sec:General Initialization And Device Operation / Device Operation}

When operating the device, each field in the device configuration
space can be changed by either the driver or the device.

Whenever such a configuration change is triggered by the device,
driver is notified. This makes it possible for drivers to
cache device configuration, avoiding expensive configuration
reads unless notified.


\subsection{Notification of Device Configuration Changes}\label{sec:General Initialization And Device Operation / Device Operation / Notification of Device Configuration Changes}

For devices where the device-specific configuration information can be
changed, a configuration change notification is sent when a
device-specific configuration change occurs.

In addition, this notification is triggered by the device setting
DEVICE_NEEDS_RESET (see \ref{sec:Basic Facilities of a Virtio Device / Device Status Field / DEVICENEEDSRESET}).

\section{Device Cleanup}\label{sec:General Initialization And Device Operation / Device Cleanup}

Once the driver has set the DRIVER_OK status bit, all the configured
virtqueue of the device are considered live.  None of the virtqueues
of a device are live once the device has been reset.

\drivernormative{\subsection}{Device Cleanup}{General Initialization And Device Operation / Device Cleanup}

A driver MUST NOT alter virtqueue entries for exposed buffers,
i.e., buffers which have been
made available to the device (and not been used by the device)
of a live virtqueue.

Thus a driver MUST ensure a virtqueue isn't live (by device reset) before removing exposed buffers.

\chapter{Virtio Transport Options}\label{sec:Virtio Transport Options}

Virtio can use various different buses, thus the standard is split
into virtio general and bus-specific sections.

\input{transport-pci.tex}
\input{transport-mmio.tex}
\input{transport-ccw.tex}
\section{Virtio Over Fabrics}\label{sec:Virtio Transport Options / Virtio Over Fabrics}

Virtio Over Fabrics (Virtio-oF) enables operations over fabrics that rely
primarily on message passing.

Virtio-oF uses a reliable connection to transmit data. The reliable
connection facilitates communication between entities playing the following roles:

\begin{itemize}
\item A Virtio-oF initiator functions as a Virtio-oF client.
The Virtio-oF initiator sends commands and associated data from the driver
to the Virtio-oF target.
\item A Virtio-oF target functions as a Virtio-oF server.
The Virtio-oF target forwards commands to the device and sends completions
and associated data back to the Virtio-oF initiator.
\end{itemize}

Virtio-oF has the following features:

\begin{itemize}
\item A Virtio-oF target is allowed to be connected by 0 or more Virtio-oF initiators.
\item A Virtio-oF initiator is allowed to connect to a single Virtio-oF target only.
A Virtio-oF device instance is a virtio device that the Virtio-oF initiator is
accessing through the Virtio-oF target.
\item There is a one-to-one mapping between the Virtio-oF queue and the reliable connection.
\item There is one, and only one, Virtio-oF control queue for a Virtio-oF device instance.
The Virtio-oF control queue is used to execute control commands,
for example, to get the Virtio Device ID.
\item There is a one-to-one mapping between virtqueue and Virtio-oF virtqueue
which executes the bulk data transport on virtio devices.
\item The arrival of data on the Virtio-oF queue indicates that a notification has arrived.
\end{itemize}


\subsection{Virtio-oF Qualified Name}\label{sec:Virtio Transport Options / Virtio Over Fabrics / Virtio-oF Qualified Name}
Virtio-oF Qualified Names (VQNs) are used to uniquely describe a Virtio-oF initiator
or a Virtio-oF target for identification.

A VQN is encoded as a string of Unicode characters with the following properties:

\begin{itemize}
\item The encoding is UTF-8 (refer to RFC 3629).
\item The characters dash('-'), dot ('.') and slash('/') are used in formatting.
\item The string is NUL terminated.
\item The maximum name is 256 bytes in length, including the NUL character.
\item There is no strict style limitation.
\end{itemize}


\subsection{Protocol Data Unit}\label{sec:Virtio Transport Options / Virtio Over Fabrics / Protocol Data Unit}
This section defines Virtio-oF Protocol Data Unit (PDU) for both the Virtio-oF control queue and Virtio-oF virtqueue.

Virtio-oF PDU is a unit of information exchanged between a Virtio-oF initiator and a Virtio-oF target:
\begin{itemize}
\item A request Virtio-oF PDU from Virtio-oF initiator to Virtio-oF target contains command and associated data, if present.
\item A response Virtio-oF PDU from Virtio-oF target to Virtio-oF initiator contains completion and associated data, if present.
\end{itemize}

\subsubsection{Stream Data Transfers}\label{sec:Virtio Transport Options / Virtio Over Fabrics / Protocol Data Unit/ Stream Data Transfers}
Stream-based (i.e. TCP/IP) Virtio-oF queue transfers Virtio-oF PDUs in a streaming fashion.

The layout in a stream:
\begin{lstlisting}
PDUx and PDUz contain a command only, PDUy contains a command and data:

     +----+     +---------+     +----+
     |PDUx|     |   PDUy  |     |PDUz|
 ... +----+ ... +----+----+ ... +----+ ...
     |CMD |     |CMD |Data|     |CMD |
     +----+     +---------+     +----+

PDUl contains completion only, PDUm and PDUn contain completion and data:

     +----+     +---------+     +---------+
     |PDUl|     |   PDUm  |     |   PDUn  |
 ... +----+ ... +----+----+ ... +----+----+ ...
     |COMP|     |COMP|Data|     |COMP|Data|
     +----+     +---------+     +---------+
\end{lstlisting}

\subsubsection{Keyed Data Transfers}\label{sec:Virtio Transport Options / Virtio Over Fabrics / Protocol Data Unit/ Keyed Data Transfers}
Message-based (i.e. RDMA) Virtio-oF queue transfers Virtio-oF PDUs in a message fashion, and uses the following structure to describe the remote data:

\begin{lstlisting}
struct virtio_of_keyed_desc {
        /* the remote address of data */
        le64 addr;
        /* the length of data */
        le32 length;
        /* the key to access the remote data */
        le32 key;
};
\end{lstlisting}

A Virtio-oF control queue supports 1 keyed descriptor, and a Virtio-oF virtqueue supports 1 or more keyed descriptors.

The PDUs of messages:
\begin{lstlisting}
PDUx contains a command only, PDUy contains a command and 1 descriptor,
and PDUz contains a command and k - j + 1 descriptors.

     +----+     +---------+     +--------------------+
     |PDUx|     |   PDUy  |     |        PDUz        |
 ... +----+ ... +----+----+ ... +----+-----+---+-----+ ...
     |CMD |     |CMD |DESC|     |CMD |DESCj|...|DESCk|
     +----+     +---------+     +----------+---+-----+

PDUl, PDUm, and PDUn contain completion only.

     +----+     +----+     +----+
     |PDUl|     |PDUm|     |PDUn|
 ... +----+ ... +----+ ... +----+ ...
     |COMP|     |COMP|     |COMP|
     +----+     +----+     +----+
\end{lstlisting}


\subsection{Commands Definition}\label{sec:Virtio Transport Options / Virtio Over Fabrics / Commands Definition}
This section defines data structures, opcodes, and status for Virtio-oF.
A Virtio-oF command is fixed to 16 bytes, and a Virtio-oF completion is fixed to 16 bytes.
Note that the reserved fields of Virtio-oF commands and completions are filled with zero.

\subsubsection{Command ID}\label{sec:Virtio Transport Options / Virtio Over Fabrics / Commands Definition / Command ID}
The Command ID (le16 type) is used to uniquely describe a Virtio-oF PDU for identification.
Generally, the Virtio-oF initiator allocates a Command ID that is unique for all in-flight commands,
and the Virtio-oF target specifies the same Command ID for completion.

Command IDs 0xff00 - 0xffff are reserved for the Virtio-oF control queue asynchronous events.
The reserved Command IDs for the Virtio-oF control queue are as follows:

\begin{tabular}{ |l|l| }
\hline
Command ID & Description \\
\hline \hline
0xfffe & Config change. Causes the initiator to generate a configuration change notification \\
\hline
0xffff & Keepalive. The initiator ignores this event \\
\hline
0xff00 - 0xfffd & Reserved \\
\hline
\end{tabular}

\subsubsection{Status}\label{sec:Virtio Transport Options / Virtio Over Fabrics / Commands Definition / Status}
The Status (le16 type) of a Virtio-oF completion is used to indicate the result of a Virtio-oF command in detail.
The status values are defined as follows:

\begin{lstlisting}
/* command executed successfully */
#define VIRTIO_OF_STATUS_SUCCESS       0x0000

/* unrecognized command, or disabled command due to unsupported feature */
#define VIRTIO_OF_STATUS_ENOCMD        0x0001
/* in-flight commands exceeded queue size */
#define VIRTIO_OF_STATUS_ECMDQUOT      0x0002

/* no such target specified by TVQN */
#define VIRTIO_OF_STATUS_ENOTGT        0x1001
/* failed to create Virtio-oF device instance */
#define VIRTIO_OF_STATUS_ENODEV        0x1002
/* rejected due to access control */
#define VIRTIO_OF_STATUS_EACLREJECTED  0x1003
/* bad Virtio-oF device instance ID */
#define VIRTIO_OF_STATUS_EBADDEV       0x1010
/* invalid VQN or mismatched TVQN/IVQN from Virtio-oF device instance */
#define VIRTIO_OF_STATUS_EBADVQN       0x1011
/* Virtio-oF virtqueue index exceeded */
#define VIRTIO_OF_STATUS_EQUEUEQUOT    0x1020
/* Virtio-oF virtqueue is already in use */
#define VIRTIO_OF_STATUS_EQUEUEBUSY    0x1021
/* Virtio-oF virtqueue size exceeded */
#define VIRTIO_OF_STATUS_EQSIZEQUOT    0x1022

/* unsupported Virtio-oF feature */
#define VIRTIO_OF_STATUS_EFEATURE      0x2000
/* unsupported Virtio-oF device instance status */
#define VIRTIO_OF_STATUS_ESTATUS       0x2010
/* unsupported Virtio-oF device instance feature */
#define VIRTIO_OF_STATUS_EDEVFEATURE   0x2020
/* unsupported offset/bytes of device configuration space */
#define VIRTIO_OF_STATUS_ECONFOFF      0x2030
/* unsupported bytes of device configuration space */
#define VIRTIO_OF_STATUS_ECONFBYTES    0x2031
/* failed to read device-readable virtqueue buffers */
#define VIRTIO_OF_STATUS_EOUTVQBUF     0x20f0
/* failed to write device-writable virtqueue buffers */
#define VIRTIO_OF_STATUS_EINVQBUF      0x20f1
\end{lstlisting}

\subsubsection{Opcodes}\label{sec:Virtio Transport Options / Virtio Over Fabrics / Commands Definition / Opcodes}
Opcodes (u16 type) of Virtio-oF Commands are as follows:

\begin{lstlisting}
#define virtio_of_op_connect               0x0000
#define virtio_of_op_disconnect            0x0001
#define virtio_of_op_keepalive             0x0002
#define virtio_of_op_get_feature           0x0004
#define virtio_of_op_set_feature           0x0005
#define virtio_of_op_get_keyed_num_descs   0x0100
#define virtio_of_op_vq                    0x0fff
#define virtio_of_op_get_vendor_id         0x1000
#define virtio_of_op_get_device_id         0x1001
#define virtio_of_op_reset_device          0x1003
#define virtio_of_op_get_status            0x1004
#define virtio_of_op_set_status            0x1005
#define virtio_of_op_get_device_feature    0x1006
#define virtio_of_op_set_driver_feature    0x1009
#define virtio_of_op_get_vq_size           0x100a
#define virtio_of_op_get_config            0x100c
#define virtio_of_op_set_config            0x100d
\end{lstlisting}

\paragraph{Connect Command}\label{sec:Virtio Transport Options / Virtio Over Fabrics / Commands Definition / Opcodes / Connect Command}

The Connect Command is used to establish a Virtio-oF queue for both the Virtio-oF control queue and Virtio-oF virtqueue, it is always the first command to execute on a Virtio-oF queue.

The structure of the Connect Command structure is as follows:
\begin{lstlisting}
struct virtio_of_command_connect {
        /* opcode is virtio_of_op_connect */
        le16 opcode;
        /* unique ID for all in-flight commands */
        le16 command_id;
        /* Virtio-oF device instance ID. 0xffff is reserved for the Virtio-oF control queue */
        le16 device_instance_id;
        /* used by Virtio-oF virtqueue only, equal to virtqueue index */
        le16 vq_index;
        /* the length of the Connect Body */
        le32 length;
#define VIRTIO_OF_CONTROL_QUEUE_SIZE 32
        /* the size of a Virtio-oF queue, 0 means the maximum queue size supported */
        le16 queue_size;
        u8 reserved[2];
};
\end{lstlisting}

The structure of the Connect Body is as follows:
\begin{lstlisting}
struct virtio_of_connect {
        /* Virtio-oF initiator VQN */
        u8 ivqn[256];
        /* Virtio-oF target VQN */
        u8 tvqn[256];
        u8 reserved[512];
};
\end{lstlisting}

The structure of the Connect Completion is as follows:
\begin{lstlisting}
struct virtio_of_completion_connect {
        le16 status;
        le16 command_id;
        /* Virtio-oF device instance ID */
        le16 device_instance_id;
        u8 reserved[10];
};
\end{lstlisting}

The Connect Command, Body, and Completion have the following usages:
\begin{itemize}
\item The Virtio-oF initiator specifies \texttt{device_instance_id} of 0xffff in the Connect command
and Virtio-oF initiator/target VQN in the Connect Body to establish the Virtio-oF control queue first.
\item The Virtio-oF target allocates any available ID for a newly created Virtio-oF device instance,
and specifies the \texttt{device_instance_id} of the new Virtio-oF device instance ID in the Connect Completion.
\item The Virtio-oF initiator specifies the Virtio-oF device instance ID and \texttt{vq_index} to establish Virtio-oF virtqueues one by one.
The Connect Body is optional for the Virtio-oF virtqueues, once a Virtio-oF virtqueue issues a Connect Command without Connect Body, the Virtio-oF virtqueue uses the same Connect Body as the Virtio-oF control queue.
\end{itemize}

\paragraph{Disconnect Command}\label{sec:Virtio Transport Options / Virtio Over Fabrics / Commands Definition / Opcodes / Disconnect Command}

The Disconnect Command is used to gracefully disconnect both the Virtio-oF control queue and Virtio-oF virtqueue. All in-flight commands are completed upon receipt of the Disconnect Completion.

The structure of the Disconnect Command structure is as follows:
\begin{lstlisting}
struct virtio_of_command_disconnect {
        /* opcode is virtio_of_op_disconnect */
        le16 opcode;
        /* unique ID for all in-flight commands */
        le16 command_id;
        u8 reserved[12];
};
\end{lstlisting}

The structure of the Disconnect Completion is as follows:
\begin{lstlisting}
struct virtio_of_completion_disconnect {
        le16 status;
        le16 command_id;
        u8 reserved[12];
};
\end{lstlisting}

The Disconnect Command and Completion have the following usages:
\begin{itemize}
\item The Virtio-oF initiator should disconnect the Virtio-oF virtqueues before the Virtio-oF control queue.
\item Once the Virtio-oF control queue is disconnected, the Virtio-oF device instance and associated resources should be destroyed.
\item If the underlying transport connection of a Virtio-oF queue is disconnected without issuing the Disconnect Command, the in-flight commands are not guaranteed to execute successfully.
\end{itemize}

\paragraph{Keepalive Command}\label{sec:Virtio Transport Options / Virtio Over Fabrics / Commands Definition / Opcodes / Keepalive Command}
The Keepalive Command is used as a health check mechanism to detect when a Virtio-oF device instance becomes unavailable, it is used for the Virtio-oF control queue only.
Once the Virtio-oF device instance becomes unavailable, the initiator should disconnect all the Virtio-oF queues and release associated resources.

The structure of the Keepalive Command structure is as follows:
\begin{lstlisting}
struct virtio_of_command_keepalive {
        /* opcode is virtio_of_op_keepalive */
        le16 opcode;
        /* unique ID for all in-flight commands */
        le16 command_id;
        u8 reserved[12];
};
\end{lstlisting}

The structure of the Keepalive Completion is as follows:
\begin{lstlisting}
struct virtio_of_completion_keepalive {
        le16 status;
        le16 command_id;
        u8 reserved[12];
};
\end{lstlisting}

\paragraph{Get Feature Command}\label{sec:Virtio Transport Options / Virtio Over Fabrics / Commands Definition / Opcodes / Get Feature Command}
The Get Feature Command is used to get feature bits of a Virtio-oF device instance for the Virtio-oF control queue only.

The structure of the Get Feature Command structure is as follows:
\begin{lstlisting}
struct virtio_of_command_get_feature {
        /* opcode is virtio_of_op_get_feature */
        le16 opcode;
        /* unique ID for all in-flight commands */
        le16 command_id;
        /* value 0x0 selects feature bits 0 to 63, 0x1 selects feature bits 64 to 127, etc */
        le32 feature_select;
        u8 reserved[8];
};
\end{lstlisting}

The structure of the Get Feature Completion is as follows:
\begin{lstlisting}
struct virtio_of_completion_get_feature {
        le16 status;
        le16 command_id;
        u8 reserved[4];
        /* feature bits */
        le64 feature;
};
\end{lstlisting}

The feature bits of a Virtio-oF device instance are as follows:
\begin{lstlisting}
/* support virtio_of_op_get_keyed_num_descs to get the maximum number of keyed descriptors */
#define VIRTIO_OF_F_KEYED_NUM_DESCS            0
\end{lstlisting}

\paragraph{Set Feature Command}\label{sec:Virtio Transport Options / Virtio Over Fabrics / Commands Definition / Opcodes / Set Feature Command}
The Set Feature Command is used to set the feature bits of a Virtio-oF device instance for the Virtio-oF control queue only.

The structure of the Set Feature Command structure is as follows:
\begin{lstlisting}
struct virtio_of_command_set_feature {
        /* opcode is virtio_of_op_set_feature */
        le16 opcode;
        /* unique ID for all in-flight commands */
        le16 command_id;
        /* value 0x0 selects feature bits 0 to 63, 0x1 selects feature bits 64 to 127, etc */
        le32 feature_select;
        /* feature bits */
        le64 feature;
};
\end{lstlisting}

The structure of the Set Feature Completion is as follows:
\begin{lstlisting}
struct virtio_of_completion_set_feature {
        le16 status;
        le16 command_id;
        u8 reserved[12];
};
\end{lstlisting}

Note: it is recommended that the
\nameref{sec:Virtio Transport Options / Virtio Over Fabrics / Commands Definition / Opcodes / Get Feature Command}
and Set Feature Command should be executed upon the establishment of the Virtio-oF control queue for feature negotiation.
However, they are allowed to be executed during the Virtio-oF control queue lifecycle.

\paragraph{Get Keyed Number Descriptors Command}\label{sec:Virtio Transport Options / Virtio Over Fabrics / Commands Definition / Opcodes / Get Keyed Number Descriptors Command}
The Get Keyed Number Descriptors Command is used to get the maximum number of keyed descriptors of Virtio-oF virtqueues, it is executed on the Virtio-oF control queue.

The structure of the Get Keyed Number Descriptors Command structure is as follows:
\begin{lstlisting}
struct virtio_of_command_get_keyed_num_descs {
        /* opcode is virtio_of_op_get_keyed_num_descs */
        le16 opcode;
        /* unique ID for all in-flight commands */
        le16 command_id;
        u8 reserved[12];
};
\end{lstlisting}

The structure of the Get Keyed Number Descriptors Completion is as follows:
\begin{lstlisting}
struct virtio_of_completion_get_keyed_num_descs {
        le16 status;
        le16 command_id;
        /* the maximum number of keyed descriptors */
        u8 descs;
        u8 reserved[11];
};
\end{lstlisting}

\paragraph{VQ Command}\label{sec:Virtio Transport Options / Virtio Over Fabrics / Commands Definition / Opcodes / VQ Command}
The VQ Command is used to transmit virtqueue buffers for Virtio-oF virtqueue only.

The structure of the VQ Command structure is as follows:
\begin{lstlisting}
struct virtio_of_command_vq {
        /* opcode is virtio_of_op_vq */
        le16 opcode;
        /* unique ID for all in-flight commands */
        le16 command_id;
        u8 reserved[4];
        /* the length of device-readable virtqueue buffers */
        le32 out_length;
        /* the length of device-writable virtqueue buffers */
        le32 in_length;
};
\end{lstlisting}

The structure of the VQ Completion is as follows:
\begin{lstlisting}
struct virtio_of_completion_vq {
        le16 status;
        le16 command_id;
        u8 reserved[4];
	/* total length of the command which was used, this is less than or equal to in_length */
        le32 length;
        /* the length of virtqueue buffers from Virtio-oF target */
        le32 in_length;
};
\end{lstlisting}

For \nameref{sec:Virtio Transport Options / Virtio Over Fabrics / Protocol Data Unit/ Stream Data Transfers},
\texttt{out_length} of virtio_of_command_vq describes the following bytes in a request Virtio-oF PDU,
and \texttt{length} of virtio_of_completion_vq describes the following bytes in a response Virtio-oF PDU.

For \nameref{sec:Virtio Transport Options / Virtio Over Fabrics / Protocol Data Unit/ Keyed Data Transfers},
there are 1 or more keyed descriptors (virtio_of_keyed_desc type) in a request Virtio-oF PDU to describe the virtqueue buffers of \texttt{out_length + in_length} bytes,
and \texttt{length} of virtio_of_completion_vq describes the number of bytes written into the device writable portion of the buffer described by the keyed descriptors.

\paragraph{Get Vendor ID Command}\label{sec:Virtio Transport Options / Virtio Over Fabrics / Commands Definition / Opcodes / Get Vendor ID Command}
The Get Vendor ID Command is used to get the Virtio Vendor ID of a Virtio-oF device instance for the Virtio-oF control queue only.

The structure of the Get Vendor ID Command structure is as follows:
\begin{lstlisting}
struct virtio_of_command_get_vendor_id {
        /* opcode is virtio_of_op_get_vendor_id */
        le16 opcode;
        /* unique ID for all in-flight commands */
        le16 command_id;
        u8 reserved[12];
};
\end{lstlisting}

The structure of the Get Vendor ID Completion is as follows:
\begin{lstlisting}
struct virtio_of_completion_get_vendor_id {
        le16 status;
        le16 command_id;
        /* Virtio Vendor ID */
        le32 vendor_id;
        u8 reserved[8];
};
\end{lstlisting}

\paragraph{Get Device ID Command}\label{sec:Virtio Transport Options / Virtio Over Fabrics / Commands Definition / Opcodes / Get Device ID Command}
The Get Device ID Command is used to get the Virtio Device ID of a Virtio-oF device instance for the Virtio-oF control queue only.

The structure of the Get Device ID Command structure is as follows:
\begin{lstlisting}
struct virtio_of_command_get_device_id {
        /* opcode is virtio_of_op_get_device_id */
        le16 opcode;
        /* unique ID for all in-flight commands */
        le16 command_id;
        u8 reserved[12];
};
\end{lstlisting}

The structure of the Get Device ID Completion is as follows:
\begin{lstlisting}
struct virtio_of_completion_get_device_id {
        le16 status;
        le16 command_id;
        /* Virtio Device ID */
        le32 device_id;
        u8 reserved[8];
};
\end{lstlisting}

\paragraph{Reset Device Command}\label{sec:Virtio Transport Options / Virtio Over Fabrics / Commands Definition / Opcodes / Reset Device Command}
The Reset Device Command is used to reset a Virtio-oF device instance for the Virtio-oF control queue only.

The structure of the Reset Device Command structure is as follows:
\begin{lstlisting}
struct virtio_of_command_reset_device {
        /* opcode is virtio_of_op_reset_device */
        le16 opcode;
        /* unique ID for all in-flight commands */
        le16 command_id;
        u8 reserved[12];
};
\end{lstlisting}

The structure of the Reset Device Completion is as follows:
\begin{lstlisting}
struct virtio_of_completion_reset_device {
        le16 status;
        le16 command_id;
        u8 reserved[12];
};
\end{lstlisting}

\paragraph{Get Status Command}\label{sec:Virtio Transport Options / Virtio Over Fabrics / Commands Definition / Opcodes / Get Status Command}
The Get Status Command is used to get the status of a Virtio-oF device instance for the Virtio-oF control queue only.

The structure of the Get Status Command structure is as follows:
\begin{lstlisting}
struct virtio_of_command_get_status {
        /* opcode is virtio_of_op_get_status */
        le16 opcode;
        /* unique ID for all in-flight commands */
        le16 command_id;
        u8 reserved[12];
};
\end{lstlisting}

The structure of the Get Status Completion is as follows:
\begin{lstlisting}
struct virtio_of_completion_get_status {
        le16 status;
        le16 command_id;
        /* status of the Virtio-oF device instance */
        le32 dev_status;
        u8 reserved[8];
};
\end{lstlisting}

\paragraph{Set Status Command}\label{sec:Virtio Transport Options / Virtio Over Fabrics / Commands Definition / Opcodes / Set Status Command}
The Set Status Command is used to set the status of a Virtio-oF device instance for the Virtio-oF control queue only.

The structure of the Set Status Command structure is as follows:
\begin{lstlisting}
struct virtio_of_command_set_status {
        /* opcode is virtio_of_op_set_status */
        le16 opcode;
        /* unique ID for all in-flight commands */
        le16 command_id;
        /* status of the Virtio-oF device instance */
        le32 status;
        u8 reserved[8];
};
\end{lstlisting}

The structure of the Set Status Completion is as follows:
\begin{lstlisting}
struct virtio_of_completion_set_status {
        le16 status;
        le16 command_id;
        u8 reserved[12];
};
\end{lstlisting}

\paragraph{Get Device Feature Command}\label{sec:Virtio Transport Options / Virtio Over Fabrics / Commands Definition / Opcodes / Get Device Feature Command}
The Get Device Feature Command is used to get virtio device feature bits of a Virtio-oF device instance for the Virtio-oF control queue only.

The structure of the Get Device Feature Command structure is as follows:
\begin{lstlisting}
struct virtio_of_command_get_device_feature {
        /* opcode is virtio_of_op_get_device_feature */
        le16 opcode;
        /* unique ID for all in-flight commands */
        le16 command_id;
        /* value 0x0 selects feature bits 0 to 63, 0x1 selects feature bits 64 to 127, etc */
        le32 feature_select;
        u8 reserved[8];
};
\end{lstlisting}

The structure of the Get Device Feature Completion is as follows:
\begin{lstlisting}
struct virtio_of_completion_get_device_feature {
        le16 status;
        le16 command_id;
        u8 reserved[4];
        /* feature bits */
        le64 feature;
};
\end{lstlisting}

\paragraph{Set Driver Feature Command}\label{sec:Virtio Transport Options / Virtio Over Fabrics / Commands Definition / Opcodes / Set Driver Feature Command}
The Set Driver Feature Command is used to set virtio driver feature bits of a Virtio-oF device instance for the Virtio-oF control queue only.

The structure of the Set Driver Feature Command structure is as follows:
\begin{lstlisting}
struct virtio_of_command_set_driver_feature {
        /* opcode is virtio_of_op_set_driver_feature */
        le16 opcode;
        /* unique ID for all in-flight commands */
        le16 command_id;
        /* value 0x0 selects feature bits 0 to 63, 0x1 selects feature bits 64 to 127, etc */
        le32 feature_select;
        /* feature bits */
        le64 feature;
};
\end{lstlisting}

The structure of the Set Driver Feature Completion is as follows:
\begin{lstlisting}
struct virtio_of_completion_set_driver_feature {
        le16 status;
        le16 command_id;
        u8 reserved[12];
};
\end{lstlisting}

\paragraph{Get VQ Size Command}\label{sec:Virtio Transport Options / Virtio Over Fabrics / Commands Definition / Opcodes / Get VQ Size Command}
The Get VQ Size Command is used to get the maximum queue size of a Virtio-oF virtqueue for the Virtio-oF control queue only.

The structure of the Get VQ Size Command structure is as follows:
\begin{lstlisting}
struct virtio_of_command_get_vq_size {
        /* opcode is virtio_of_op_get_vq_size */
        le16 opcode;
        /* unique ID for all in-flight commands */
        le16 command_id;
        /* the Virtio-oF virtqueue index */
        le16 vq_index;
        u8 reserved[10];
};
\end{lstlisting}

The structure of the Get VQ Size Completion is as follows:
\begin{lstlisting}
struct virtio_of_completion_get_vq_size {
        le16 status;
        le16 command_id;
        /* size of Virtio-oF virtqueues */
        le16 size;
        u8 reserved[10];
};
\end{lstlisting}

Note: the Virtio-oF control queue has a fixed queue size, the Virtio-oF initiator can specify a smaller Virtio-oF virtqueue size by
\nameref{sec:Virtio Transport Options / Virtio Over Fabrics / Commands Definition / Opcodes / Connect Command}.

\paragraph{Get Config Command}\label{sec:Virtio Transport Options / Virtio Over Fabrics / Commands Definition / Opcodes / Get Config Command}
The Get Config Command is used to get the device configuration space of a Virtio-oF device instance for the Virtio-oF control queue only.

The structure of the Get Config Command structure is as follows:
\begin{lstlisting}
struct virtio_of_command_get_config {
        /* opcode is virtio_of_op_get_config */
        le16 opcode;
        /* unique ID for all in-flight commands */
        le16 command_id;
        /* offset of device configuration space */
        le16 offset;
        /* bytes to get, available values: 1, 2, 4, 8 */
        u8 bytes;
        u8 reserved[9];
};
\end{lstlisting}

The structure of the Get Config Completion is as follows:
\begin{lstlisting}
struct virtio_of_completion_get_config {
        le16 status;
        le16 command_id;
        /* generation count for the device configuration space */
        le32 generation;
        /* v is u8, u16, u32, or u64, then: v = (typeof v)le64_to_cpu(config) */
        le64 config;
};
\end{lstlisting}

\paragraph{Set Config Command}\label{sec:Virtio Transport Options / Virtio Over Fabrics / Commands Definition / Opcodes / Set Config Command}
The Set Config Command is used to set the device configuration space of a Virtio-oF device instance for the Virtio-oF control queue only.

The structure of the Set Config Command structure is as follows:
\begin{lstlisting}
struct virtio_of_command_set_config {
        /* opcode is virtio_of_op_set_config */
        le16 opcode;
        /* unique ID for all in-flight commands */
        le16 command_id;
        /* offset of device configuration space */
        le16 offset;
        /* bytes to get, available values: 1, 2, 4, 8 */
        u8 bytes;
        u8 reserved;
        /* v is u8, u16, u32, or u64, then: config = cpu_to_le64((u64)v) */
        le64 config;
};
\end{lstlisting}

The structure of the Set Config Completion is as follows:
\begin{lstlisting}
struct virtio_of_completion_set_config {
        le16 status;
        le16 command_id;
        u8 reserved[12];
};
\end{lstlisting}

\paragraph{Config Change Completion}\label{sec:Virtio Transport Options / Virtio Over Fabrics / Commands Definition / Opcodes / Config Change Completion}
A Config Change Completion is used as a configuration change notification of a Virtio-oF device instance for the Virtio-oF control queue only.
Note: only a single outstanding Config Change Completion is allowed, the configuration change notification is suppressed until the Virtio-oF initiator issues a
\nameref{sec:Virtio Transport Options / Virtio Over Fabrics / Commands Definition / Opcodes / Get Config Command}.

The structure of the Config Change Completion is as follows:
\begin{lstlisting}
struct virtio_of_completion_config_change {
        le16 status;
        /* command_id is 0xfffe */
        le16 command_id;
        /*  generation count for the device configuration space */
        le32 generation;
        u8 reserved[8];
};
\end{lstlisting}

\paragraph{Keepalive Completion}\label{sec:Virtio Transport Options / Virtio Over Fabrics / Commands Definition / Opcodes / Keepalive Completion}
A Keepalive Completion is used as a health check mechanism to detect when a Virtio-oF initiator becomes unavailable, it is used by a Virtio-oF device instance for the Virtio-oF control queue only.
Once the Virtio-oF initiator becomes unavailable, the Virtio-oF target should destroy the Virtio-oF device instance and associated resources.


\subsection{Transport Binding}\label{sec:Virtio Transport Options / Virtio Over Fabrics / Transport Binding}

\subsubsection{TCP/IP}\label{sec:Virtio Transport Options / Virtio Over Fabrics / Transport Binding / TCP IP}
Virtio-oF supports both TCP/IPv4 and TCP/IPv6, it uses
\nameref{sec:Virtio Transport Options / Virtio Over Fabrics / Protocol Data Unit/ Stream Data Transfers}.
A TCP/IP based Virtio-oF target uses port 8549 (CRC-16/ARC of "Virtio") by default.

\subsubsection{TLS TCP/IP}\label{sec:Virtio Transport Options / Virtio Over Fabrics / Transport Binding / TLS TCP IP}
Virtio-oF supports both TLS v1.2 and TLS v1.3 over the underlying TCP/IPv4 and TCP/IPv6, it uses
\nameref{sec:Virtio Transport Options / Virtio Over Fabrics / Protocol Data Unit/ Stream Data Transfers}.
A TLS TCP/IP based Virtio-oF target uses port 8549 (CRC-16/ARC of "Virtio") by default.

\subsubsection{RDMA}\label{sec:Virtio Transport Options / Virtio Over Fabrics / Transport Binding / RDMA}
Virtio-oF supports RDMA, it uses
\nameref{sec:Virtio Transport Options / Virtio Over Fabrics / Protocol Data Unit/ Keyed Data Transfers}.
A RoCEv2 based Virtio-oF target uses UDP port 8549 (CRC-16/ARC of "Virtio") by default, and a iWARP based Virtio-oF target uses TCP port 8549 by default.

RDMA uses the following operations:
\begin{itemize}
\item The Virtio-oF initiator sends a PDU by RDMA SEND.
\item The Virtio-oF target reads the remote data by RDMA READ.
\item The Virtio-oF target writes the remote data by RDMA WRITE.
\item The Virtio-oF target sends a PDU by RDMA SEND.
\end{itemize}


\subsection{Device Initialization}\label{sec:Virtio Transport Options / Virtio Over Fabrics / Device Initialization}
\begin{enumerate}
\item The Virtio-oF control queue must be established first by connecting
from the Virtio-oF initiator to the Virtio-oF target and issuing a
\nameref{sec:Virtio Transport Options / Virtio Over Fabrics / Commands Definition / Opcodes / Connect Command}
to create a Virtio-oF device instance.
\item Optionally, The Virtio-oF initiator issues a
\nameref{sec:Virtio Transport Options / Virtio Over Fabrics / Commands Definition / Opcodes / Get Feature Command}
to discover the features offered by the Virtio-oF device instance.
After feature negotiation, the Virtio-oF initiator also issues a
\nameref{sec:Virtio Transport Options / Virtio Over Fabrics / Commands Definition / Opcodes / Set Feature Command}
to finalize features.
\item The Virtio-oF initiator issues a
\nameref{sec:Virtio Transport Options / Virtio Over Fabrics / Commands Definition / Opcodes / Get Device ID Command}
and
\nameref{sec:Virtio Transport Options / Virtio Over Fabrics / Commands Definition / Opcodes / Get Vendor ID Command}
for virtio driver.
\item During virtio device initialization, the Virtio-oF initiator issues a
\nameref{sec:Virtio Transport Options / Virtio Over Fabrics / Commands Definition / Opcodes / Get Status Command}
and
\nameref{sec:Virtio Transport Options / Virtio Over Fabrics / Commands Definition / Opcodes / Set Status Command}.
\item Optionally, The Virtio-oF initiator issues a
\nameref{sec:Virtio Transport Options / Virtio Over Fabrics / Commands Definition / Opcodes / Get VQ Size Command}
to get the size of a Virtio-oF virtqueue.
\item The Virtio-oF initiator issues a
\nameref{sec:Virtio Transport Options / Virtio Over Fabrics / Commands Definition / Opcodes / Connect Command}
to establish a Virtio-oF virtqueue.
\item The Virtio-oF initiator creates Virtio-oF virtqueues one by one, then issues a
\nameref{sec:Virtio Transport Options / Virtio Over Fabrics / Commands Definition / Opcodes / VQ Command}
to supply virtqueue buffers to the Virtio-oF device instance on Virtio-oF virtqueues.
\end{enumerate}


\chapter{Device Types}\label{sec:Device Types}

On top of the queues, config space and feature negotiation facilities
built into virtio, several devices are defined.

The following device IDs are used to identify different types of virtio
devices.  Some device IDs are reserved for devices which are not currently
defined in this standard.

Discovering what devices are available and their type is bus-dependent.

\begin{tabular} { |l|c| }
\hline
Device ID  &  Virtio Device    \\
\hline \hline
0          & reserved (invalid) \\
\hline
1          &   network device     \\
\hline
2          &   block device     \\
\hline
3          &      console       \\
\hline
4          &  entropy source    \\
\hline
5          & memory ballooning (traditional)  \\
\hline
6          &     ioMemory       \\
\hline
7          &       rpmsg        \\
\hline
8          &     SCSI host      \\
\hline
9          &   9P transport     \\
\hline
10         &   mac80211 wlan    \\
\hline
11         &   rproc serial     \\
\hline
12         &   virtio CAIF      \\
\hline
13         &  memory balloon    \\
\hline
16         &   GPU device       \\
\hline
17         &   Timer/Clock device \\
\hline
18         &   Input device \\
\hline
19         &   Socket device \\
\hline
20         &   Crypto device \\
\hline
21         &   Signal Distribution Module \\
\hline
22         &   pstore device \\
\hline
23         &   IOMMU device \\
\hline
24         &   Memory device \\
\hline
25         &   Sound device \\
\hline
26         &   file system device \\
\hline
27         &   PMEM device \\
\hline
28         &   RPMB device \\
\hline
29         &   mac80211 hwsim wireless simulation device \\
\hline
30         &   Video encoder device \\
\hline
31         &   Video decoder device \\
\hline
32         &   SCMI device \\
\hline
33         &   NitroSecureModule \\
\hline
34         &   I2C adapter \\
\hline
35         &   Watchdog \\
\hline
36         &   CAN device \\
\hline
38         &   Parameter Server \\
\hline
39         &   Audio policy device \\
\hline
40         &   Bluetooth device \\
\hline
41         &   GPIO device \\
\hline
42         &   RDMA device \\
\hline
43         &   Camera device \\
\hline
44         &   ISM device \\
\hline
45         &   SPI master \\
\hline
\end{tabular}

Some of the devices above are unspecified by this document,
because they are seen as immature or especially niche.  Be warned
that some are only specified by the sole existing implementation;
they could become part of a future specification, be abandoned
entirely, or live on outside this standard.  We shall speak of
them no further.

\input{device-types/net/description.tex}
\input{device-types/blk/description.tex}
\input{device-types/console/description.tex}
\input{device-types/entropy/description.tex}
\input{device-types/balloon/description.tex}
\input{device-types/scsi/description.tex}
\input{device-types/gpu/description.tex}
\input{device-types/input/description.tex}
\input{device-types/crypto/description.tex}
\input{device-types/vsock/description.tex}
\input{device-types/fs/description.tex}
\input{device-types/rpmb/description.tex}
\input{device-types/iommu/description.tex}
\input{device-types/sound/description.tex}
\input{device-types/mem/description.tex}
\input{device-types/i2c/description.tex}
\input{device-types/scmi/description.tex}
\input{device-types/gpio/description.tex}
\input{device-types/pmem/description.tex}

\chapter{Reserved Feature Bits}\label{sec:Reserved Feature Bits}

Currently these device-independent feature bits are defined:

\begin{description}
  \item[VIRTIO_F_INDIRECT_DESC (28)] Negotiating this feature indicates
  that the driver can use descriptors with the VIRTQ_DESC_F_INDIRECT
  flag set, as described in \ref{sec:Basic Facilities of a Virtio
Device / Virtqueues / The Virtqueue Descriptor Table / Indirect
Descriptors}~\nameref{sec:Basic Facilities of a Virtio Device /
Virtqueues / The Virtqueue Descriptor Table / Indirect
Descriptors} and \ref{sec:Packed Virtqueues / Indirect Flag: Scatter-Gather Support}~\nameref{sec:Packed Virtqueues / Indirect Flag: Scatter-Gather Support}.
  \item[VIRTIO_F_EVENT_IDX(29)] This feature enables the \field{used_event}
  and the \field{avail_event} fields as described in
\ref{sec:Basic Facilities of a Virtio Device / Virtqueues / Used Buffer Notification Suppression}, \ref{sec:Basic Facilities of a Virtio Device / Virtqueues / The Virtqueue Used Ring} and \ref{sec:Packed Virtqueues / Driver and Device Event Suppression}.


  \item[VIRTIO_F_VERSION_1(32)] This indicates compliance with this
    specification, giving a simple way to detect legacy devices or drivers.

  \item[VIRTIO_F_ACCESS_PLATFORM(33)] This feature indicates that
  the device can be used on a platform where device access to data
  in memory is limited and/or translated. E.g. this is the case if the device can be located
  behind an IOMMU that translates bus addresses from the device into physical
  addresses in memory, if the device can be limited to only access
  certain memory addresses or if special commands such as
  a cache flush can be needed to synchronise data in memory with
  the device. Whether accesses are actually limited or translated
  is described by platform-specific means.
  If this feature bit is set to 0, then the device
  has same access to memory addresses supplied to it as the
  driver has.
  In particular, the device will always use physical addresses
  matching addresses used by the driver (typically meaning
  physical addresses used by the CPU)
  and not translated further, and can access any address supplied to it by
  the driver. When clear, this overrides any platform-specific description of
  whether device access is limited or translated in any way, e.g.
  whether an IOMMU may be present.
  \item[VIRTIO_F_RING_PACKED(34)] This feature indicates
  support for the packed virtqueue layout as described in
  \ref{sec:Basic Facilities of a Virtio Device / Packed Virtqueues}~\nameref{sec:Basic Facilities of a Virtio Device / Packed Virtqueues}.
  \item[VIRTIO_F_IN_ORDER(35)] This feature indicates
  that all buffers are used by the device in the same
  order in which they have been made available.
  \item[VIRTIO_F_ORDER_PLATFORM(36)] This feature indicates
  that memory accesses by the driver and the device are ordered
  in a way described by the platform.

  If this feature bit is negotiated, the ordering in effect for any
  memory accesses by the driver that need to be ordered in a specific way
  with respect to accesses by the device is the one suitable for devices
  described by the platform. This implies that the driver needs to use
  memory barriers suitable for devices described by the platform; e.g.
  for the PCI transport in the case of hardware PCI devices.

  If this feature bit is not negotiated, then the device
  and driver are assumed to be implemented in software, that is
  they can be assumed to run on identical CPUs
  in an SMP configuration.
  Thus a weaker form of memory barriers is sufficient
  to yield better performance.
  \item[VIRTIO_F_SR_IOV(37)] This feature indicates that
  the device supports Single Root I/O Virtualization.
  Currently only PCI devices support this feature.
  \item[VIRTIO_F_NOTIFICATION_DATA(38)] This feature indicates
  that the driver passes extra data (besides identifying the virtqueue)
  in its device notifications.
  See \ref{sec:Basic Facilities of a Virtio Device / Driver notifications}~\nameref{sec:Basic Facilities of a Virtio Device / Driver notifications}.

  \item[VIRTIO_F_NOTIF_CONFIG_DATA(39)] This feature indicates that the driver
  uses the data provided by the device as a virtqueue identifier in available
  buffer notifications.
  As mentioned in section \ref{sec:Basic Facilities of a Virtio Device / Driver notifications}, when the
  driver is required to send an available buffer notification to the device, it
  sends the virtqueue index to be notified. The method of delivering
  notifications is transport specific.
  With the PCI transport, the device can optionally provide a per-virtqueue value
  for the driver to use in driver notifications, instead of the virtqueue index.
  Some devices may benefit from this flexibility by providing, for example,
  an internal virtqueue identifier, or an internal offset related to the
  virtqueue index.

  This feature indicates the availability of such value. The definition of the
  data to be provided in driver notification and the delivery method is
  transport specific.
  For more details about driver notifications over PCI see \ref{sec:Virtio Transport Options / Virtio Over PCI Bus / PCI-specific Initialization And Device Operation / Available Buffer Notifications}.

  \item[VIRTIO_F_RING_RESET(40)] This feature indicates
  that the driver can reset a queue individually.
  See \ref{sec:Basic Facilities of a Virtio Device / Virtqueues / Virtqueue Reset}.

  \item[VIRTIO_F_ADMIN_VQ(41)] This feature indicates that the device exposes one or more
  administration virtqueues.
  At the moment this feature is only supported for devices using
  \ref{sec:Virtio Transport Options / Virtio Over PCI
	  Bus}~\nameref{sec:Virtio Transport Options / Virtio Over PCI Bus}
	  as the transport and is reserved for future use for
	  devices using other transports (see
	  \ref{drivernormative:Basic Facilities of a Virtio Device / Feature Bits}
	and
	\ref{devicenormative:Basic Facilities of a Virtio Device / Feature Bits} for
	handling features reserved for future use.

\end{description}

\drivernormative{\section}{Reserved Feature Bits}{Reserved Feature Bits}

A driver MUST accept VIRTIO_F_VERSION_1 if it is offered.  A driver
MAY fail to operate further if VIRTIO_F_VERSION_1 is not offered.

A driver SHOULD accept VIRTIO_F_ACCESS_PLATFORM if it is offered, and it MUST
then either disable the IOMMU or configure the IOMMU to translate bus addresses
passed to the device into physical addresses in memory.  If
VIRTIO_F_ACCESS_PLATFORM is not offered, then a driver MUST pass only physical
addresses to the device.

A driver SHOULD accept VIRTIO_F_RING_PACKED if it is offered.

A driver SHOULD accept VIRTIO_F_ORDER_PLATFORM if it is offered.
If VIRTIO_F_ORDER_PLATFORM has been negotiated, a driver MUST use
the barriers suitable for hardware devices.

If VIRTIO_F_SR_IOV has been negotiated, a driver MAY enable
virtual functions through the device's PCI SR-IOV capability
structure.  A driver MUST NOT negotiate VIRTIO_F_SR_IOV if
the device does not have a PCI SR-IOV capability structure
or is not a PCI device.  A driver MUST negotiate
VIRTIO_F_SR_IOV and complete the feature negotiation
(including checking the FEATURES_OK \field{device status}
bit) before enabling virtual functions through the device's
PCI SR-IOV capability structure.  After once successfully
negotiating VIRTIO_F_SR_IOV, the driver MAY enable virtual
functions through the device's PCI SR-IOV capability
structure even if the device or the system has been fully
or partially reset, and even without re-negotiating
VIRTIO_F_SR_IOV after the reset.

A driver SHOULD accept VIRTIO_F_NOTIF_CONFIG_DATA if it is offered.

\devicenormative{\section}{Reserved Feature Bits}{Reserved Feature Bits}

A device MUST offer VIRTIO_F_VERSION_1.  A device MAY fail to operate further
if VIRTIO_F_VERSION_1 is not accepted.

A device SHOULD offer VIRTIO_F_ACCESS_PLATFORM if its access to
memory is through bus addresses distinct from and translated
by the platform to physical addresses used by the driver, and/or
if it can only access certain memory addresses with said access
specified and/or granted by the platform.
A device MAY fail to operate further if VIRTIO_F_ACCESS_PLATFORM is not
accepted.

If VIRTIO_F_IN_ORDER has been negotiated, a device MUST use
buffers in the same order in which they have been available.

A device MAY fail to operate further if
VIRTIO_F_ORDER_PLATFORM is offered but not accepted.
A device MAY operate in a slower emulation mode if
VIRTIO_F_ORDER_PLATFORM is offered but not accepted.

It is RECOMMENDED that an add-in card based PCI device
offers both VIRTIO_F_ACCESS_PLATFORM and
VIRTIO_F_ORDER_PLATFORM for maximum portability.

A device SHOULD offer VIRTIO_F_SR_IOV if it is a PCI device
and presents a PCI SR-IOV capability structure, otherwise
it MUST NOT offer VIRTIO_F_SR_IOV.

\section{Legacy Interface: Reserved Feature Bits}\label{sec:Reserved Feature Bits / Legacy Interface: Reserved Feature Bits}

Transitional devices MAY offer the following:
\begin{description}
\item[VIRTIO_F_NOTIFY_ON_EMPTY (24)] If this feature
  has been negotiated by driver, the device MUST issue
  a used buffer notification if the device runs
  out of available descriptors on a virtqueue, even though
  notifications are suppressed using the VIRTQ_AVAIL_F_NO_INTERRUPT
  flag or the \field{used_event} field.
\begin{note}
  An example of a driver using this feature is the legacy
  networking driver: it doesn't need to know every time a packet
  is transmitted, but it does need to free the transmitted
  packets a finite time after they are transmitted. It can avoid
  using a timer if the device notifies it when all the packets
  are transmitted.
\end{note}
\end{description}

Transitional devices MUST offer, and if offered by the device
transitional drivers MUST accept the following:
\begin{description}
\item[VIRTIO_F_ANY_LAYOUT (27)] This feature indicates that the device
  accepts arbitrary descriptor layouts, as described in Section
  \ref{sec:Basic Facilities of a Virtio Device / Virtqueues / Message Framing / Legacy Interface: Message Framing}~\nameref{sec:Basic Facilities of a Virtio Device / Virtqueues / Message Framing / Legacy Interface: Message Framing}.

\item[UNUSED (30)] Bit 30 is used by qemu's implementation to check
  for experimental early versions of virtio which did not perform
  correct feature negotiation, and SHOULD NOT be negotiated.
\end{description}
